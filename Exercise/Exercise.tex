\documentclass[a4paper,12pt]{article}

\usepackage{url}
\usepackage{amsmath,amssymb}
\usepackage{setspace}
\usepackage{fontspec}             % for Thai unicode characters
\XeTeXlinebreaklocale "th"
\XeTeXlinebreakskip = 0pt plus 1pt
\usepackage[Latin,Thai]{ucharclasses}
\defaultfontfeatures{Mapping=tex-text} 
%\setmainfont{TeX Gyre Termes}				% Free Times
%\setsansfont{TeX Gyre Heros}				% Free Helvetica
%\setmonofont{TeX Gyre Cursor}				% Free Courier

%% For Windows, can  use these existing fonts
\setmainfont{Times New Roman}	
\setsansfont{Arial}						% Helvetica substitute
\setmonofont{Courier New}				

\newfontfamily{\thaifont}[Scale=MatchUppercase,Mapping=tex-text]{TH Sarabun New:script=thai}

\usepackage{listings}
\lstset{
basicstyle=\ttfamily,
numbers=left,
numberstyle=\small,
breaklines=true,
xleftmargin=.1\linewidth,
frame=single,
columns=fullflexible,
captionpos=b,
showstringspaces=false
}

\onehalfspacing

\setTransitionTo{Thai}{\thaifont}
\setTransitionFrom{Thai}{\normalfont}

\title{Fourier Transform}
\author{Wolfram MathWorld}
\date{}

\begin{document}
\maketitle
ข้อความข้างล่างนี้ตัดมาจาก \cite{WolframForier} ให้ทดลองเขียน source code ของ \LaTeX~เอง เพื่อสร้างเอกสารนี้ โครงของเอกสารคือไฟล์ Exercise.tex

\section{การอ้างถึง การแทรกรูป และการใส่เชิงอรรถ}
\begin{figure}[h]
\caption{ตัวอย่างการแปลงฟูเรียร์}
\label{Fig:TransformExample}
\end{figure}

\section{การเขียนสมการ และการสร้างข้อย่อย}
ตัวอย่างข้างล่างนี้มีทั้งการเขียนคำสั่งคณิตศาสตร์แทรกระหว่างข้อความ และการกำหนดสภาพแวดล้อมคณิตศาสตร์ มีการใช้ cases ในการกำหนดค่าให้ฟังก์ชัน และการแทรกข้อความลงในสภาพแวดล้อมคณิตศาสตร์ ลองสังเกตดูว่าฟอนต์ที่ใช้ในสภาพแวดล้อมคณิตศาสตร์กับสภาพแวดล้อมข้อความปกตินั้นต่างกัน ผู้เขียนเอกสารพึงระวังเสมอเมื่อต้องการอ้างถึงตัวแปรต่าง ๆ

ส่วนเงื่อนไขด้านล่างใช้การสร้างข้อย่อยแบบมีเลขข้อ

\section{การเขียนสมการ (เพิ่มเติม) และการสร้างตาราง}
\label{Sec:Table}
ตัวอย่างข้างล่างนี้เป็นการใช้สมการหลายบรรทัดและมีการจัดตำแหน่งให้ตรงกัน ในที่นี้จัดตำแหน่งของ $\equiv$ ให้ตรงกับ $=$ ในบรรทัดถัดมา นอกจากนี้ยังมีการใช้สัญลักษณ์พิเศษ $\star$ การใส่ bar เหนือตัวแปร ($\bar{f}$) รวมถึงฟอนต์พิเศษสำหรับ $\mathcal{F}, \mathbf{x}, \mathbf{k}$ และ $\mathbb{R}$ ด้วย สัญลักษณ์พิเศษเหล่านี้ต้องใช้ package amsmath และ amssymb ผู้ที่ต้องใช้สัญลักษณ์ทางคณิตศาสตร์เป็นประจำควรจดจำได้ว่าต้องใช้ฟอนต์แบบใดกับตัวแปรหรือสัญลักษณ์ที่ต้องการ

\section{ข้อสังเกตอื่นๆ}
Caption ใต้ภาพยังเป็น Figure อยู่ ซึ่งเกิดจากคลาส article ที่ใช้นั้นกำหนดไว้เป็นภาษาอังกฤษ หากต้องการปรับให้เป็นภาษาไทย สามารถตั้งให้เป็นคำที่ต้องการเองได้โดยใช้คำสัง renewcommand เช่น หากสั่ง \lstinline|\renewcommand{\figurename}{Fig.}| Caption ของรูปจะเปลี่ยนจากค่าเริ่มต้นเติม (Figure) ไปเป็น Fig. เป็นต้น หัวข้อของเอกสารอ้างอิงด้านล่างก็เช่นเดียวกัน

ในคลาส chula นั้นมีคำสั่งกำหนดคำให้แล้วทั้งภาษาไทยและภาษาอังกฤษ ซึ่งสามารถเลือกใช้ได้โดยกำหนด option ของคลาสเป็น thaithesis หรือ engthesis

ตารางในข้อ \ref{Sec:Table} นั้นไม่มีชื่อตารางกำกับ และยังยาวเกินกว่าขอบเขตของข้อความที่กำหนด ให้ผู้เรียนลองศึกษาการเขียนตารางรูปแบบอื่นๆ ซึ่งสามารถกำหนดขอบเขตของตารางให้อยู่ในกรอบที่กำหนดด้วยตนเอง

\bibliographystyle{plain}
\bibliography{Exercise}
\end{document}