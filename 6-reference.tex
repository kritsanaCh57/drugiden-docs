\usepackage{biblatex}
\addbibresource{MathCS.bib}


\chapter{สรุปและข้อเสนอแนะ}
จากขั้นตอนการศึกษาและเริ่มพัฒนาแอปพลิเคชันค้นหายาเพื่อคุณ โดยผ่านกระบวนการต่างๆ 
ได้แก่ การวิเคราะห์แบะออกแบบระบบ การพัฒนาโปรแกรมและการทดสอบโปรแกรม จนสิ้นสุดกระบวนการซึ่งสามารถสรุปผลของโครงงานได้ดังนี้

\begin{itemize}[label={--}]
	\item สรุปความสามารถของระบบ
  \item ปัญหาและอุปสรรคในการพัฒนา
  \item แนวทางการพัฒนาต่อ
\end{itemize}

\section{สรุปความสามารถของระบบ}
แอปพลิเคชันค้นหายาเพื่อคุณ ได้ดำเนินการออกแบบสร้างระบบงานและทำการทดสอบระบบงาน สามารถสรุปผลได้ดังนี้ โดยจะแบ่งเป็น 2 ส่วน คือ ส่วนแอปพลิเคชัน และส่วนเว็บเซอร์วิส ดังนี้
\begin{enumerate}
  \item ส่วนแอปพลิเคชัน
  \begin{itemize}[label={--}]
    \item สามารถค้นหายาเม็ดได้ทั้งแบบทั้วไปและแบบขั้นสูง
    \item	สามารถถ่ายรูปภาพเพื่อการพิสูจน์เอกลักษณ์ยาเม็ดได้
    \item	สามารถดูรายละเอียดของยาได้
    \item	สามารถบันทึกบุ๊กมาร์กรายการยาได้
  \end{itemize}
  \item ส่วนเว็บเซอร์วิส
  \begin{itemize}[label={--}]
    \item สามารถติดต่อกับฐานข้อมูลการพิสูจน์เอกลักษณ์ยาเม็ดหรือแคปซูล คณะเภสัชศาสตร์ มหาวิทยาลัยอุบลราชธานี
    \item สามารถจำแนกรูปทรงของเม็ดยาทั้ง 6 รูปทรง ได้แก่ สี่เหลี่ยม สามเหลี่ยม วงกลม วงกลม หกเหลี่ยม และแปดเหลี่ยม
    \item สามารถระบุเอกลักษณะสีของยาเม็ดได้ และระบุความขนาดของเม็ดยาได้ ด้วยการประมวลผลภาพ
    \item สามารถป้องกันการร้องขอทรัพยากรจากบุคคลอื่นที่ไม่ใช่แอปพลิเคชันค้นหายาเพื่อคุณ
  \end{itemize}
\end{enumerate}

\section{ปัญหาและอุปสรรคในการพัฒนา}
  \begin{enumerate}
    \item ปัญหา : การติดตั้งเว็บเซอร์วิสที่เครื่องเซิฟเวอร์นอกเนื่องจากการพัฒนา จำเป็นต้องติดตั้งโปรแกรมเสริมจำนวนมากและเวอร์ชันของโปรแกรมไม่เหมือนกัน ทำให้ไม่สามารถเริ่มการทำงานของเว็บเซอร์วิสได้ 
    
    แนวทางการแก้ไข : การพัฒนาเว็บเซอร์วิสไว้ใน Docker container เพื่อลดปัญหาการติดตั้งโปรแกรมเสริมและการไม่เข้ากันของเวอร์ชันของโปรแกรม เพียงแค่เครื่องเซิฟเวอร์ติดตั้ง Docker
    \item ปัญหา : การหาข้อมูลรูปภาพรูปทรงต่างๆ สำหรับฝึกและสอนแบบจำลอง รูปทรงหกเหลี่ยมและรูปทรงแปดเหลี่ยมในฐานข้อมูลการพิสูจน์เอกลักษณ์ยาเม็ดหรือแคปซูล คณะเภสัชศาสตร์ มหาวิทยาลัยอุบลราชธานี มีจำนวนน้อย
    
    แนวทางการแก้ไข : ค้นหารูปภาพจากแหล่งที่มาอื่นมาเพิ่ม
    \item ปัญหา : รูปภาพที่ได้มาจากฐานข้อมูลการพิสูจน์เอกลักษณ์ยาเม็ดหรือแคปซูล และแหล่งที่มาอื่น ยังไม่สามารถนำมาใช้งานได้ต้องทำการทำความสะอาดข้อมูลรูปภาพเพื่อฝึกสอนและทดสอบแบบจำลอง 
    
    แนวทางการแก้ไข : เขียนโค้ดภาษา Javascript สำหรับการทำความสะอาดรูปภาพจำนวนมาก โดยแยกตามรูปทรงและแบ่งข้อมูลสำหรับฝึกสอนและทดสอบแบบจำลอง
    
  \end{enumerate}

\section{แนวทางการพัฒนาต่อ}
\begin{enumerate}
	\item การปรับแบบจำลองให้รองรับกับรูปทรงอื่นๆ ที่นอกเนื่องจากรูปทรงที่ผู้พัฒนาใช้ในการสร้างแบบจำลอง เช่น รูปหัวใจ รูปสัตว์ต่างๆ เป็นต้น
  \item พัฒนาถ่ายรูปและประมวลผลแบบ Real Time เพื่อการพิสูจน์เอกลักษณ์ที่สะดวกสบายขึ้น
\end{enumerate}

% \cite{JSON:Web:Token}

% \cite{Routing}

% \cite{Express}

% \cite{Contours}

% \cite{Canny:Edge}

% \cite{WebView}

% \cite{MySQL}

% \cite{Nginx}

% \cite{Docker}

% \cite{Navigating:Lifecycle:Events}

% \cite{Native:Application}

% \cite{Web:Application}

% \cite{Drug:Identification:Database}

% \cite{TCP/IP}

% \cite{Docker:Hub}

% \cite{Ionic:Framework}

% \cite{Ubuntu}

% \cite{Nearest:Neighbors}

% \cite{RandomForestClassifier}

% \cite{ratio}

% \citeauthor{bib1}

% \citeauthor{bib2}

% \citeauthor{bib3}

% \citeauthor{bib4}

% \citeauthor{bib5}

% \citeauthor{bib6}

% \citeauthor{bib7}

% \citeauthor{bib8}

% \citeauthor{bib9}

% \citeauthor{bib10}

% \citeauthor{bib11}

% \citeauthor{bib12}

% \citeauthor{bib13}

% \citeauthor{bib14}

% \citeauthor{bib15}

% \citeauthor{bib16}

% \citeauthor{bib17}

% \citeauthor{bib18}

% \citeauthor{bib19}

% \citeauthor{bib20}

\newpage
\cite{bib1}

\cite{bib2}

\cite{bib3}

\cite{bib4}

\cite{bib5}

\cite{bib6}

\cite{bib7}

\cite{bib8}

\cite{bib9}

\cite{bib10}

\cite{bib11}

\cite{bib12}

\cite{bib13}

\cite{bib14}

\cite{bib15}

\cite{bib16}

\cite{bib17}

\cite{bib18}

\cite{bib19}

\cite{bib20}
