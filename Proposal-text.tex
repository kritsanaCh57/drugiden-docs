\documentclass[tikz]{standalone}

\usepackage[round,semicolon]{natbib}
\usepackage{pgfgantt}           % for gantt chart

\begin{document}
\title{แบบเสนอหัวข้อโครงงาน รายวิชา 2301399 \\ Project Proposal \\ ภาคการศึกษาต้น ปีการศึกษา 2558}

\makeatletter
\begin{center}
    {\Large \@title \par}
\end{center}
\makeatother

\begin{tabular}{p{4.5cm}p{9cm}}
ชื่อโครงงาน (ภาษาไทย) & การใช้~\LaTeX ~สำหรับเรียงพิมพ์วิทยานิพนธ์ภาษาไทยและภาษาอังกฤษโดยใช้รูปแบบของจุฬาลงกรณ์มหาวิทยาลัย \\
ชื่อโครงงาน (ภาษาอังกฤษ)	& Introduction to \LaTeX for typesetting Thai and English thesis with Chulalongkorn University thesis format \\
อาจารย์ที่ปรึกษา & อาจารย์ ดร.บุญฤทธิ์ อินทิยศ \\
ผู้ดำเนินการ	& อาจารย์ ดร.ฑิตยา หวานวารี \\
\end{tabular}
    \vspace*{0.4cm}
    \hrule width 15 cm height 0.0 cm depth 0.025 cm

\section{หลักการและเหตุผล}
\LaTeX~เป็นซอฟต์แวร์สำหรับเรียงพิมพ์ที่เหมาะสำหรับงานด้านคณิตศาสตร์ สามารถเรียงพิมพ์สมการและสัญลักษณ์ต่าง ๆ ได้สะดวกและสวยงาม จึงได้รับความนิยมมากในกลุ่มนักคณิตศาสตร์ วิทยาศาสตร์ รวมถึงงานอื่น ๆ ที่จำเป็นต้องอ้างอิงสมการคำนวณต่าง ๆ แต่ลักษณะการใช้งานของ \LaTeX~ นั้นต่างกับซอฟต์แวร์ประมวลคำทั่วไป ซึ่งมักจะเห็นผลลัพธ์ของการจัดหน้าต่าง ๆ ในขณะที่พิมพ์ จึงจำเป็นต้องเรียนรู้เพิ่มเติมอยู่บ้าง โครงงานนี้จึงจัดทำขึ้นเพื่อเป็นแนวทางสำหรับผู้เริ่มต้นใช้ \LaTeX~ในการทำเอกสาร โดยสามารถนำไปใช้เป็นแบบร่างสำหรับทำเอกสารประกอบโครงงาน รวมถึงวิทยานิพนธ์ของจุฬาลงกรณ์มหาวิทยาลัยได้
\section{วัตถุประสงค์}
เพื่อแนะนำการใช้ {\normalfont \LaTeX} เบื้องต้น และเตรียมรูปแบบโครงร่างเอกสารสำหรับนำไปปรับใช้ในเล่มโครงงานหรือวิทยานิพนธ์ฉบับสมบูรณ์ได้
\section{ขอบเขตของโครงงาน}
\begin{enumerate}
    \item แนะนำการพิมพ์เอกสารในภาษาไทย และภาษาอังกฤษ ผ่านการเรียงพิมพ์โดย {\normalfont Xe\LaTeX}
    \item ได้ตัวอย่างรูปเล่มรายงานที่ใช้กับคลาส chula.cls และ chulanat.bst ซึ่งเป็นรูปแบบการเรียงพิมพ์วิทยานิพนธ์ของจุฬาลงกรณ์มหาวิทยาลัย
\end{enumerate}
\section{ขั้นตอนการทำงาน}
โครงงานนี้ประกอบด้วยการรวบรมเอกสารการคู่มือการใช้ {\normalfont \LaTeX} เขียนเอกสารคำแนะนำประกอบโครงการ ปรับแต่งรูปแบบของคลาส chula.cls เพื่อให้ใช้กับโครงงานระดับปริญญาบัณฑิตได้ และจัดอบรมการใช้ {\normalfont \LaTeX} สำหรับผู้เริ่มต้น

\section{ระยะเวลาการทำงาน}
จากขั้นตอนการทำงานข้างต้น สามารถเขียน Gantt chart ได้ดังนี้

\begin{ganttchart}[
    hgrid,
    vgrid,
    time slot format=isodate-yearmonth,
    compress calendar,
    x unit=.9cm,
    y unit chart=1.5cm,
    bar label node/.append style={align=left, text width=4cm}
    ]{2014-08}{2015-07}
    \gantttitlecalendar{year, month} \\
    \ganttbar{1. รวบรวมเอกสารคู่มือการใช้ {\normalfont \LaTeX}}{2014-08}{2014-10} \\
    \ganttbar{2. เขียนเอกสารแนะนำประกอบโครงการ}{2014-09}{2015-03-31} \\
    \ganttbar{3. ปรับแต่งคลาส chula.cls}{2014-12}{2015-06-30} \\
    \ganttbar{4. จัดอบรมการใช้ {\normalfont \LaTeX}}{2015-07-01}{2015-07-31}
\end{ganttchart}

รายละเอียดเพิ่มเติมเกี่ยวกับการวาด Gantt chart สามารถอ่านได้ที่ \citep{pgfgantt}
\section{ประโยชน์ที่คาดว่าจะได้รับ}
นิสิตและผู้ที่สนใจสามารถนำโครงร่างเอกสารนี้ไปปรับใช้และเขียนเอกสารประกอบโครงงานส่งได้
\section{อุปกรณ์และเครื่องมือที่ใช้}
\begin{enumerate}
    \item ขุดซอฟต์แวร์สำหรับเรียงพิมพ์ เช่น Mik\TeX, \TeX Live, Mac\TeX โดยใช้ Xe\LaTeX
    \item ฟอนต์ TH Sarabun New สำหรับภาษาไทย
    \item ฟอนต์ TeX Gyre Termes, TeX Gyre Heros และ TeX Gyre Cursor สำหรับภาษาอังกฤษ
    \item คลาส chula.cls
    \item รูปแบบรายการอ้างอิง chulanat.bst
\end{enumerate}
\end{document}