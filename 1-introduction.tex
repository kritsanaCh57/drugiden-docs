\chapter{บทนำ}

\section{ที่มาและเหตุผล }
ปัญหาจากการได้รับประทานยาปลอมหรือยาที่ถูกเพิกถอนทะเบียนเป็นปัญหาใหญ่ในปัจจุบัน เพื่อป้องกันปัญหานี้ก่อนจะมีการใช้ยาใด ๆ ที่ไม่รู้จัก จึงต้องมีการตรวจสอบยาให้ละเอียดก่อน แต่ในปัจจุบันการพิสูจน์เอกลักษณ์ยาเม็ดหรือแคปซูลเป็นไปได้ยาก เนื่องมาจากความคล้ายคลึงกันของยาเม็ดหรือแคปซูลในหลายชนิด ทั้งขนาด สี รูปทรง ชื่อยา เป็นต้น ดังนั้นการพิสูจน์เอกลักษณ์ยาเม็ดหรือแคปซูลจำเป็นต้องอาศัยเภสัชกรหรือผู้เชี่ยวชาญที่มีประสบการณ์และความสามารถในการจดจำในการพิสูจน์เอกลักษณ์ยาเม็ด แต่การจดจำลักษณะของเม็ดยาจำนวนมากและข้อมูลรายละเอียดของยาแต่ละชนิดจำเป็นต้องใช้การอ้างอิงข้อมูลในการพิสูจน์เอกลักษณ์ยาเม็ดที่เชื่อถือได้ เช่น ค้นหาในเว็บไซต์พิสูจน์ยาโดยเฉพาะที่ได้รับการรับรองจากหน่วยงานราชการเป็นต้น 

หน่วยข้อมูลยาและสุขภาพ คณะเภสัชศาสตร์ มหาวิทยาลัยอุบลราชธานี ได้ริเริ่มทำงานวิจัยที่รวบรวมข้อมูลได้แก่ ชื่อการค้า ชื่อสามัญทางยา รูปแบบผลิตภัณฑ์ รูปร่างลักษณะ สัญลักษณ์หรือตัวอักษรบนยาเม็ดหรือแคปซูล สี ขนาดของยาเม็ดและแคปซูล บริษัทผู้ผลิต และบริษัทผู้จำหน่ายโดยเริ่มจากการจัดทำฐานข้อมูลการพิสูจน์เอกลักษณ์ของยาเม็ดและแคปซูลก่อนและได้สร้างเว็บไซต์ Drug Identification Database [13] เพื่อใช้ในการเป็นฐานข้อมูลในการพิสูจน์เอกลักษณ์ยา 

เนื่องจากการใช้เว็บไซต์ Drug Identification Database ยังไม่สามารถรองรับการใช้งานผ่านโทรศัพท์สมาร์ทโฟนและเพื่อความสะดวกของการใช้งาน ผู้พัฒนาจึงได้ทำโครงการแอปพลิเคชันค้นหายาเพื่อคุณที่ผู้ใช้งานสามารถสืบค้นหารายละเอียดยาได้และสามารถถ่ายรูปยาเม็ดเพื่อการพิสูจน์เอกลักษณ์ของยาเม็ดหรือแคปซูลได้โดยไม่ต้องอาศัยเภสัชกรหรือผู้เชียวชาญในการจดจำในการพิสูจน์เอกลักษณ์ยาและสามารถตอบโต้กับผู้ใช้งานได้ทันที โดยใช้ความเป็นโทรศัพท์สมาร์ทโฟนที่สามารถใช้งานได้ที่ทุกที่ทุกเวลา

\section{วัตถุประสงค์}
    
    \begin{enumerate}
        \item เพื่อพัฒนาแอปพลิเคชันที่สามารถการสืบค้นข้อมูลและการพิสูจน์เอกลักษณ์เม็ดยาหรือแคปซูลได้
        
        \item เพื่อให้การพิสูจน์เอกลักษณ์ของยาเม็ดหรือแคปซูลสามารถทำงานและดูผลลัพธ์ได้ด้วยผู้ใช้งานเอง
        
    \end{enumerate}
\section{ขอบเขตของโครงงาน}
    
    \begin{enumerate}
        \item แอปพลิเคชันสามารถทำงานบนระบบทำงานบนโทรศัพท์สมาร์ทโฟนระบบปฏิบัติการแอนดรอยด์ได้
        \item แอปพลิเคชันสามารถสืบค้นหาข้อมูลยาเม็ดหรือแคปซูลได้ โดยใช้ฐานข้อมูลพิสูจน์เอกลักษณ์ยาเม็ดหรือแคปซูลในประเทศไทย
        \item แอปพลิเคชันสามารถถ่ายรูปภาพและประมวลผลรูปภาพเพื่อพิสูน์เอกลักษณ์ของยาเม็ดและแคปซูลได้
    \end{enumerate}
\section{ประโยชน์ที่คาดว่าจะได้รับ}
    
    เป็นแอปพลิเคชั่นที่ใช้ในการพิสูจเอกลักษณ์ยาเม็ดหรือแคปซูล การให้ข้อมูลผู้ใช้งานเกี่ยวกับข้อมูลการผลิตยาเม็ดหรือแคปซูล และเพื่อให้เป็นเครื่องมือในการเฝ้าระวังปัญหาที่เกิดจากการได้รับยาปลอมหรือยาที่ถูกเพิกถอนทะเบียน
\section{เครื่องมือที่ใช้ในการพัฒนา}
    
    \begin{enumerate}
        \item ด้านฮาร์ดแวร์
        
        \begin{enumerate}
            \item ความต้องการของระบบสำหรับโทรศัพท์สมาร์ทโฟน
            
            \begin{itemize}
                \item ระบบปฏิบัติการแอนดรอยด์ 4.1 ขึ้นไป หรือ ระบบปฏิบัติการไอโอเอส 7 ขึ้นไป
                \item จอแสดงผล 768 x 1366 พิกเซล
                \item หน่วยความจำหลัก 4GB
                \item หน่วยความจำรอง 8GB
                \item ความละเอียดของกล้างถ่ายรูป 12 MP
            \end{itemize}    
            \item ความต้องการของระบบสำหรับเครื่องคอมพิวเตอร์ส่วนบุคคล
            
            \begin{itemize}
                \item ระบบปฏิบัติการ windows 7 ขึ้นไป 
                \item หน่วยประมวลผล Intel Pentium 4, 3GHz ขึ้นไป หรือ AMD Athlon 64 3000 ขึ้นไป
                \item หน่วยความจำหลัก 4GB
                \item หน่วยความจำลอง 128GB
            \end{itemize}
        \end{enumerate}
        \item ด้านซอฟต์แวร์
        
        \begin{enumerate}
            \item Node.js ใช้สำหรับสร้างเว็บเซอร์วิสที่ติดต่อกับระหว่างผู้ใช้งานกับฐานข้อมูลด้วย RESTful API
            \item IONIC 3 Framework เป็นเครื่องมือที่ใช้ในการพัฒนา Mobile Application แบบ Hybrid (Hybrid Mobile App คือการพัฒนาแอปพลิเคชันครั้งเดียวแล้วสามารถทำงานได้หลาย Platform) 
            \item OpenCV ย่อมาจาก Open Source Computer Vision ซึ่งเป็นไลบรารี่ที่รวบรวมฟังก์ชั่นต่างๆสำหรับการประมวลผลภาพและคอมพิวเตอร์วิทัศนศาสตร์เอาไว้เป็นจำนวนมาก ไลบรารี่นี้อยู่ภายใต้ใบอนุญาต BSD ซึ่งเราสามารถใช้ได้ฟรีทั้งทางด้านการศึกษาและทางการค้า นอกจากนั้น OpenCV ยังมีอินเตอร์เฟสที่หลากหลายรองรับการพัฒนาโปรแกรมบนภาษาโปรแกรมต่างๆ เช่น C/C++, Python, Java, Javascript เป็นต้น และยังสามารถรันได้ทั้งบน Window, Linux, Android, และ Mac
            \item Android SDK (Software Development Kit) เป็นชุดเครื่องมือที่เอาไว้สำหรับพัฒนาแอปพลิเคชัน Android OS
        \end{enumerate}
    \end{itemize}
\section{ขั้นตอนการดำเนินงาน}
\begin{table}[H]
    \centering
    \caption{ขั้นตอนการดำเนินงาน}
    \begin{tabular}{| p{5cm} | c| c| c| c| c| c| c| c| c| c|}
    \hline
    {\centering\setstretch{1.0}แผนดำเนินงานโครงงาน} & \rot{สิงหาคม 2560} & \rot{กันยายน 2560} & \rot{ตุลาคม 2560} & \rot{พฤศจิกายน 2560} & \rot{ธันวาคม 2560} & \rot{มกราคม 2561} & \rot{กุมภาพันธ์ 2561} & \rot{มีนาคม 2561} & \rot{เมษายน 2561} & \rot{พฤษภาคม 2561} \\ \hline
    {\setstretch{1.0}วิเคราะห์ระบบและออกแบบ}                        & \cellcolor{blue!25} &  &  &  &  &  &  &  &  &  \\ \hline
    {\setstretch{1.0}ออกแบบหน้าต่างแอปพลิเคชั่นและเว็บเซอร์วิส}          &  & \cellcolor{blue!25} & \cellcolor{blue!25} & \cellcolor{blue!25} & \cellcolor{blue!25} &  &  &  &  &  \\ \hline
    {\setstretch{1.0}การสร้างระบบระบุตัวตนด้วย JWT ให้กับเว็บเซอร์วิส}      &  &  &  & \cellcolor{blue!25} & \cellcolor{blue!25} & \cellcolor{blue!25} &  &  &  &  \\ \hline
    {\setstretch{1.0}ศึกษา OpenCV และดำเนินการเขียนการประมวลผลรูปภาพ} &  &  &  &  & \cellcolor{blue!25} & \cellcolor{blue!25} & \cellcolor{blue!25} &  &  &  \\ \hline
    {\setstretch{1.0}ดำเนินการพัฒนาแอปพลิเคชัน}                      &  &  &  &  & \cellcolor{blue!25} & \cellcolor{blue!25} & \cellcolor{blue!25} & \cellcolor{blue!25} &  &  \\ \hline
    {\setstretch{1.0}ดำเนินการเขียนเอกสาร}                           &  &  &  &  &  &  &  & \cellcolor{blue!25} & \cellcolor{blue!25} & \cellcolor{blue!25} \\ \hline
    \end{tabular}
\end{table}


\AtBeginDocument{\showtimer}