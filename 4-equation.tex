\chapter{การสร้างระบบ}
การสร้างระบบงานเป็นส่วนหนึ่งของการพัฒนาแอปพลิเคชันค้นหายาเพื่อคุณและเว็บเซอร์วิส ซึ่งจะอธิบายการทำงานของแต่ละส่วน ดังนี้
\begin{itemize}
	\item การค้นหายาแบบทั่วไป
  	\item การค้นหายาแบบขั้นสูง
  	\item การถ่ายรูปภาพ
  	\item	การเรียกใช้งานเว็บเซอร์วิส
  	\item	การติดต่อฐานข้อมูลของเว็บเซอร์วิส
  	\item	การสร้างและการตรวจสอบ JSON Web Token
	\item	การถอดรหัสรูปภาพ BASE64
	\item	การ routing ของเว็บเซอร์วิส
  	\item	การประมวลผลภาพ
\end{itemize}

\section{การค้นหายาแบบทั่วไป}

	เมื่อผู้ใช้งานกรอกข้อความในแถบค้นหาทั่วไป 
	ระบบจะร้องขอการค้นหากับเว็บเซอร์วิสและส่งรายการการค้นหากลับมาแสดงที่หน้ารายการค้นหา 
	แสดงดังภาพที่ \ref{Fig:searchGeneral}
	
	\begin{figure}[H]
		{\setstretch{1.0}\begin{lstlisting}
onInput()  {        
	this.offsetStart  =  1;        
	this.offsetEnd  =  20;        
	this.content.resize();        
	if  (this.searchText)  {            
		this.searchProvider.ganeralSearch(this.searchText.trim(),  this.offsetStart,  this.offsetEnd).then(data  =>  {     
			this.drugs  =  data['results'];                
			this.drugsLength  =  data['length'];          
			if  (data['length']  ==  0)  {          
				this.presentToast();          
			}      
		}).catch((error)  =>  {          
			let  alert  =  this.alertCtrl.create({                    
				title: Error,  
				buttons:  ['close']          
			});          
			alert.present();     
		});    
	}   
}		
		\end{lstlisting}}
		\caption{การค้นหายาแบบทั่วไป}
		\label{Fig:searchGeneral}
	\end{figure}
	จากภาพที่ \ref{Fig:searchGeneral} อธิบายการทำงานของการค้นหายาแบบทั่วไป ได้ดังนี้
	\begin{itemize}[label={--}]
		\item บรรทัดที่  2-3 	กำหนดค่าของตัวแปร สำหรับกำหนดช่วงของรายการที่ต้องการ
		\item บรรทัดที่  4	เป็นคำสั่งเพื่อปรับขนาดของหน้า content
		\item บรรทัดที่  5	เป็นเงื่อนไข ถ้าหากตัวแปรว่างหรือไม่เก็บข้อความใดๆ จะไม่มีการเรียกใช้งานเซอร์วิส
		\item บรรทัดที่  6	เป็นการใช้งานคลาส SearchProvider ฟังก์ชัน ganeralSearch เพื่อร้องขอการค้นหากับเว็บเซอร์วิส 
		\item บรรทัดที่  7	เก็บรายการของการค้นหายาเป็นอาเรย์ในตัวแปร
		\item บรรทัดที่  8	เก็บความยาวของรายการการค้นหายาในตัวแปร
		\item บรรทัดที่  9-10	เงื่อนไขใช้ตรวจสอบความยาวของรายการยา ถ้าหากเท่ากับ 0 แสดงว่าไม่พบเจอรายการ และจะแสดงข้อความขึ้นมาบนหน้าจอ
		\item บรรทัดที่  13-17	ถ้าหากการร้องขอรายการค้นหาเกิดข้อผิดพลาด จะแสดงข้อความขึ้นมาบนหน้าจอ
	\end{itemize}

\section{การค้นหายาแบบขั้นสูง}
	เมื่อผู้ใช้งานกรอกรายละเอียดยาจากหน้าค้นหา 
	ระบบจะร้องขอการค้นหากับเว็บเซอร์วิสและส่งรายการการค้นหากลับมาแสดงที่หน้ารายการค้นหา 
	แสดงดังภาพที่ \ref{Fig:searchAdvance}

	\begin{figure}[H]
		{\setstretch{1.0}\begin{lstlisting}
advanceSearch()  {        
	let  params  =  this.advanceModel();        
	let  loading  =  this.loadingCtrl.create({            
		content:   'Searhing...'        
	});        
    loading.present();        
	this.content.resize();        
	this.offsetStart  =  1;        
	this.offsetEnd  =  20;        
	this.searchProvider.advanceSearch(params,  this.offsetStart,  this.offsetEnd).then(data  =>  {          
		this.drugs  =  data['results'];            
		this.drugsLength  =  data['length'];            
		loading.dismiss();            
		if  (data['length']  ==  0)  {                
			this.presentToast();            
		}        
   }).catch((error)  =>  {            
        loading.dismiss();            
        let  alert  =  this.alertCtrl.create({  
            title: Error  
	    });            
	    alert.present();        
	});    
}    
		\end{lstlisting}}
		\caption{การค้นหายาแบบขั้นสูง}
		\label{Fig:searchAdvance}
	\end{figure}
	จากภาพที่ \ref{Fig:searchAdvance} อธิบายการทำงานของการค้นหาแบบขั้นสูง ได้ดังนี้
	\begin{itemize}[label={--}]
		\item บรรทัดที่ 2	กำหนดตัวแปรสำหรับเก็บข้อมูลโมเดลของ
		\item บรรทัดที่ 3-6	แสดงการ loading 
		\item บรรทัดที่ 7	เป็นคำสั่งเพื่อปรับขนาดของหน้า content
		\item บรรทัดที่ 8-9	กำหนดค่าของตัวแปร สำหรับกำหนดช่วงของรายการที่ต้องการ
		\item บรรทัดที่ 10	เป็นการใช้งานคลาส SearchProvider ฟังก์ชัน advanceSearch เพื่อร้องขอการค้นหากับเว็บเซอร์วิส
		\item บรรทัดที่ 11	เก็บรายการของการค้นหายาเป็นอาเรย์ในตัวแปร
		\item บรรทัดที่ 12	เก็บความยาวของรายการการค้นหายาในตัวแปร
		\item บรรทัดที่ 13	ปิดการแสดงการ loading
		\item บรรทัดที่ 14-15	เงื่อนไขใช้ตรวจสอบความยาวของรายการยา ถ้าหากเท่ากับ 0 แสดงว่าไม่พบเจอรายการ และจะแสดงข้อความขึ้นมาบนหน้าจอ
		\item บรรทัดที่ 17-21	ถ้าหากการร้องขอรายการค้นหาเกิดข้อผิดพลาด จะแสดงข้อความขึ้นมาบนหน้าจอ
		
	\end{itemize}


\section{การถ่ายรูปภาพ}
	ในหน้าการถ่ายรูปภาพเพื่อพิสูจน์เอกลักษณ์ยา 
	เมื่อผู้ใช้งานกดเปิดกล้อง 
	ระบบจะทำการเรียกใช้งานไลบรารี่ Camera ของ Cordova 
	เพื่อทำเรียกใช้งานกล้องของอุปกรณ์ 
	แสดงดังภาพที่ \ref{Fig:camera}

	\begin{figure}[H]
		{\setstretch{1.0}\begin{lstlisting}
options:  CameraOptions  =   {        
    quality:  70,  
	destinationType:  this.camera.DestinationType.DATA_URL,  
    encodingType:  this.camera.EncodingType.JPEG,  
    mediaType:  this.camera.MediaType.PICTURE    
}    
takePicture()   {              
    this.camera.getPicture(this.options).then((imageData)   =>   {                      
        this.imagebase64 = 'data:image/jpeg;base64,'   +   imageData;              
    },     (err)   =>   {                      
        console.log(err);              
    });      
}      
		\end{lstlisting}}
		\caption{การถ่ายรูปภาพ}
		\label{Fig:camera}
	\end{figure}
	จากภาพที่ \ref{Fig:camera} อธิบายการทำงานของการถ่ายรูปภาพ ได้ดังนี้
	\begin{itemize}[label={--}]
		\item บรรทัดที่ 1-6	กำหนด options ของการถ่ายรูป คุณภาพของรูป ชนิดของรูปภาพ 
		\item บรรทัดที่ 7-13	เป็นฟังก์ชันการถ่ายรูปภาพ เมื่อการถ่ายรูปสำเร็จ รูปภาพจะถูกเก็บในรูปแบบ base64 ในตัวแปรชื่อ imagebase64
		\item บรรทัดที่ 8	เปิดการทำงานของกล้องถ่ายรูปบนอุปกรณ์
		\item บรรทัดที่ 9 	กำหนดตัวแปร imagebase64 สำหรับเก็บรูปภาพในรูปแบบ base64 ที่ได้จากการถ่ายรูป
	\end{itemize}

\section{การเรียกใช้งานเว็บเซอร์วิส}
	การเรียกใช้งานเว็บเซอร์วิสด้วยการใช้ไลบรารี HTTP ของ Cordova 
	แสดงดังภาพที่ \ref{Fig:callService}

	\begin{figure}[H]
		{\setstretch{1.0}\begin{lstlisting}
setOption()  {        
    let  headers  =  new  Headers;        
    headers.set('api-key',  this.apiToken);        
    headers.append('Content-type',  'application/json')  this.opt  =  new  RequestOptions({            
        headers:  headers        
    });    
}    
ganeralSearch(input:  string,  offsetStart:  any,  offsetEnd:  any)  {        
    if  (offsetStart  ==  -1  ||  offsetEnd  ==  -1)  {            
	    offsetStart  =  1,  offsetEnd  =  10;        
	}        
	return  new  Promise((resolve,  reject)  =>  {            
		this.http.get(`${this.apiEndPoint}/drugs/search?input=${input}&offsetStart=${offsetStart}&offsetEnd=${offsetEnd}`,  this.opt).map(res  =>  res.json()).subscribe(data  =>  {                
			resolve(data);            
		},  error  =>  {                
			reject(error);            
		});        
    });    
}    
		\end{lstlisting}}
		\caption{การเรียกใช้งานเว็บเซอร์วิสs}
		\label{Fig:callService}
	\end{figure}
	จากภาพที่ \ref{Fig:callService} อธิบายการทำงานของการใช้งานเว็บเซอร์วิส ได้ดังนี้
	
	\begin{itemize}[label={--}]
		\item บรรทัดที่ 1-7	เป็นฟังก์ชันการกำหนดค่าเริ่มต้นของ Header โดยจะกำหนดรูปแบบและส่ง api-token 
		\item บรรทัดที่ 8-19	เป็นฟังก์ชันสำหรับติดต่อกับเว็บเซอร์วิส โดยยกตัวอย่างการค้นหาแบบทั่วไป
		\item บรรทัดที่ 12-17	เป็นการเรียกใช้งาน HTTP แบบ GET จะร้องขอไปที่อยู่ (URL) ของเว็บเซอร์วิส
		
	\end{itemize}

\section{การติดต่อฐานข้อมูลของเว็บเซอร์วิส}
	การติดต่อกับฐานข้อมูลการพิสูจน์เอกลักษณํยาเม็ดหรือแคปซูลของคณะเภสัชศาสตร์ มหาวิทยาวัยอุบลราชธานี 
	ใช้ไลบรารี่ของ nodejs ชื่อ mysql 
	แสดงดังภาพที่ \ref{Fig:connectDB}

	\begin{figure}[H]
		{\setstretch{1.0}\begin{lstlisting}
const  mysql  =  require('mysql');    
const  configDB  =  require('./config');    
var  pool  =  mysql.createPool(configDB.info);    
pool.getConnection(function(err,  connection)  {        
	if  (err)  throw  err;        
	console.log('Connected',  new  Date());    
});    
		\end{lstlisting}}
		\caption{การติดต่อฐานข้อมูลของเว็บเซอร์วิส}
		\label{Fig:connectDB}
	\end{figure}
	จากภาพที่ \ref{Fig:connectDB} อธิบายการทำงานของการติดต่อฐานข้อมูลของเว็บเซอร์วิส ได้ดังนี้
	\begin{itemize}[label={--}]
		\item บรรทัดที่ 1	กำหนดตัวแปรชื่อ mysql สำหรับเก็บไลบรารีชื่อว่า mysql
		\item บรรทัดที่ 2	กำหนดตัวแปรชื่อ configDB สำหรับเก็บข้อมูลการ key-value เพื่อใช้ติดต่อกับฐานข้อมูล
		\item บรรทัดที่ 3	กำหนดตัวแปรชื่อ pool สำหรับเก็บข้อมูลการสร้างการเชื่อมต่อกับฐานข้อมูล
		\item บรรทัดที่ 4-7	เป็นการเรียกใช้งานฟังก์ชันชื่อว่า getConnection เพื่อติดต่อกับฐานข้อมูล 
	\end{itemize}


\section{การสร้างและการตรวจสอบ JSON Web Token}
	นำ JSON Web Token มาช่วยในการระบุตัวตนก่อนให้ใช้งานกับเว็บเซอร์วิส 
	โดยใช้ไลบรารี่ของ nodejs ชื่อว่า jsonwebtoken 
	แสดงดังภาพที่ \ref{Fig:jsonwebtokenMaker}
	\begin{figure}[H]
		{\setstretch{1.0}\begin{lstlisting}
var  jwt  =  require('jsonwebtoken');    
var  fs  =  require('fs');    
const  uuid  =  require('uuid/v4');    
module.exports  =   {        
	sign:   function(req,  res,  next)  {            
		var  data  =   {                
			id:  req.query.id,  
			uuid:  uuid()            
		};            
		var  cert  =  fs.readFileSync('./auth/keys/private.key');            
		jwt.sign(data,  cert,   {                
			algorithm:   'RS256',  
			expiresIn:   "120d"            
		},  function(err,  token)  {                
			if  (err)  {                    
				console.log('Token error ',  err.message);                    
				return  res.status(405).send({                        
					err:  err                    
				});                
			}                
			req.query.token  =  token;                
			return  res.status(200).send({                    
				token:  req.query.token                
			});            
		});        
	},verify:   function(req,  res,  next)  {            
		var  cert  =  fs.readFileSync('./auth/keys/public.pem');            
		jwt.verify(req.headers['api-key'],  cert,   {                
			algorithms:  ['RS256']            
		},  function(err,  payload)  {                
			if  (err)  {                    
				return  res.status(401).send({                        
					err:  err                    
				});                
			}                
			req.query.payload  =  payload;                
			next();           
		});        
	}  
};    
		\end{lstlisting}}
		\caption{การสร้างและการตรวจสอบ JSON Web Token}
		\label{Fig:jsonwebtokenMaker}
	\end{figure}
	จากภาพที่ \ref{Fig:jsonwebtokenMaker} อธิบายการทำงานของการสร้างและการตรวจสอบ JSON Web Token ได้ดังนี้
	\begin{itemize}[label={--}]
		\item บรรทัดที่ 1	กำหนดตัวแปรชื่อ jwt สำหรับเรียกใช้งานไลบารีชื่อว่า jsonwebtoken 
		\item บรรทัดที่ 2	กำหนดตัวแปรชื่อ js สำหรับเรียกใช้งานไลบารีชื่อว่า js เพื่อการเขียนไฟล์
		\item บรรทัดที่ 5-26	เป็นฟังก์ชันการสร้าง jwt โดยจะใช้ fs อ่านไฟล์ชื่อว่า private.pem เพื่อใช้สำหรับการเข้ารหัส กำหนดรูปแบบและอายุการใช้งานของ jwt
		\item บรรทัดที่ 27-40	เป็นฟังก์ชันการตรวจสอบความถูกต้องของ jwt โดยจะใช้ fs อ่านไฟล์ชื่อว่า public.pem เพื่อใช้สำหรับการถอดรหัสและตรวจสอบ jwt
	\end{itemize}


\section{การถอดรหัสรูปภาพ BASE64}
	ในการส่งรูปภาพจากเครื่องผู้ใช้งานมายังเว็บเซอร์วิส 
	จะถูกส่งรูปภาพมาในรูปแบบ base64 
	แสดงดังรูปภาพที่ \ref{Fig:decodeBase64}

	\begin{figure}[H]
		{\setstretch{1.0}\begin{lstlisting}
var  fs  =  require('fs');    
const  uuid  =  require('uuid/v4');    
module.exports  =   {        
	decode64:   function(req,  res,  next)  {            
		var  body  =  req.body;            
		var  base64Data  =  body.imageBase64.replace(/^data:image\/jpeg;base64,/,  "");            
		let  filename  =  `${uuid()}.jpeg`;            
		let  filePath  =  `./assets/images/${filename}`;            
		fs.writeFile(filePath,  base64Data,  'base64',  function(err)  {                
			if  (err)  {                    
				console.log('Error ',  err.message);                    
				return  res.status(500).send(err.message);                
			}                
			req.query.filename  =  filename;                
			req.query.filePath  =  filePath;                
			console.log(req.query.filePath);                
			next();            
		});       
	}    
};    
		\end{lstlisting}}
		\caption{การถอดรหัสรูปภาพ BASE64}
		\label{Fig:decodeBase64}
	\end{figure}
	จากภาพที่ \ref{Fig:decodeBase64} อธิบายการทำงานของการถอดรหัสรูปภาพ BASE64 ได้ดังนี้
	\begin{itemize}[label={--}]
		\item บรรทัดที่ 1	กำหนดตัวแปรชื่อ js สำหรับเรียกใช้งานไลบารีชื่อว่า js เพื่อการเขียนไฟล์
		\item บรรทัดที่ 2	กำหนดตัวแปรชื่อ uuid สำหรับเรียกใช้งานไลบารีชื่อว่า uuid/v4 เพื่อการสร้างชื่อไฟล์
		\item บรรทัดที่ 4-19	เป็นฟังก์ชันการถอดรหัสรูปภาพ BASE64
		\item บรรทัดที่ 5	กำหนดตัวแปรชื่อ body สำหรับเก็บค่าจาก req.body
		\item บรรทัดที่ 6	กำหนดตัวแปรชื่อ base64Data สำหรับเก็บข้อมูลรูปภาพจาก body 
		\item บรรทัดที่ 7	กำหนดตัวแปรชื่อ filename สำหรับเก็บชื่อไฟล์ภาพ จากการเรียกใช้งานฟังก์ชัน uuid ของไลบรารี่ uuid/v4
		\item บรรทัดที่ 9 	เรียกใช้งานฟังก์ชัน writeFile จาก fs สำหรับการเขียนไฟล์
		\item บรรทัดที่ 14-15	เก็บค่า filename กับ filePate ไว้ใน req.query
		\item บรรทัดที่ 17	เรียกฟังก์ชัน next() เพื่อทำงานฟังก์ชันต่อไป
	\end{itemize}

\section{การ routing ของเว็บเซอร์วิส}
การ routing ของเว็บเซอร์วิสใช้ไบรารี่ของ nodejs ชื่อ express 
สำหรับเป็นเว็บเซอร์วิสติดต่อกับฐานข้อมูลและประมวลผลภาพ 
แสดงดังภาพที่ \ref{Fig:expressRouting}

	\begin{figure}[H]
		{\setstretch{1.0}\begin{lstlisting}
var   app   =   require('express')();      
var   bodyParser   =   require('body-parser');      
var   cors   =   require('cors');      
var   POST   =   process.env.PORT   ||   5000;    
var   con   =   require('./db/connectDB')   
var   jwt   =   require('./auth/auth');      
var   decode64   =   require('./base64/base64');     
var   opencv    =    require('./openCV/svmTest');        
app.use(bodyParser.json({              
	limit:     '5mb'      
}));      
app.use(bodyParser.urlencoded({              
	limit:     '5mb',  
	extended:   true      
}));      
app.use(cors());      
app.use('/drugs',   jwt.verify,   con.checkAuth);    
app.get('/token/sign',   jwt.sign);      
app.get('/token/verify',   jwt.verify,   jwt.verified);    
app.get('/drugs/id',   con.getDrug);      
app.get('/drugs/search',   con.genaralSearchDrugs);      
app.get('/drugs/search/advance',   con.advanceSearchDrugs);      
app.get('/drugs/dimg',   con.getDimg);      
app.get('/drugs/bookmarks',   con.getBookmarks)  
app.get('/ddl/color',   con.ddlColor);      
app.get('/ddl/dgroup',   con.ddlDgroup);      
app.get('/ddl/drtype',   con.ddlDrtype);      
app.get('/ddl/dshape',   con.ddlDshape);      
app.get('/ddl/dsize',   con.ddlDsize);      
app.get('/ddl/dstatus',   con.ddlDstatus);      
app.get('/ddl/dtype',   con.ddlDtype);      
app.get('/ddl/shapetype',    con.ddlShapetype);      
app.post('/api/drugs/upload',   decode64.decode64,   opencv.getSizeAndColors,   opencv.predict);      
var   server   =   app.listen(POST,   function()   {              
	console.log('express is running on port:'   +   POST);      
});      
		\end{lstlisting}}
		\caption{การ routing ของเว็บเซอร์วิส}
		\label{Fig:expressRouting}
	\end{figure}
	จากภาพที่ \ref{Fig:expressRouting} อธิบายการทำงานของการ routing ของเว็บเซอร์วิส ได้ดังนี้
	\begin{itemize}[label={--}]
		\item บรรทัดที่ 1	เรียกใช้งานไลบรารี Express และเก็บไว้ในตัวแปรชื่อ app เป็นไลบรารีหลักในการทำงานของ Web application framework
		\item บรรทัดที่ 2	เรียกใช้งานไลบรารี bodyparser และเก็บไว้ในตัวแปรชื่อ bodyparser เป็น Middleware ทำหน้าที่เป็นตัวกรอง requeset 
		\item บรรทัดที่ 3	เรียกใช้งานไลบรารี cors และเก็บไว้ในตัวแปรชื่อ cors เป็นกลไลที่ทำให้เว็บเซอร์วิสสามารถอนุญาตการร้องขอทรัพยากรจากเครื่องผู้ใช้งาน
		\item บรรทัดที่ 4	กำหนดตัวแปรชื่อ PORT 
		\item บรรทัดที่ 5	เรียกใช้งานการเชื่อมต่อกับฐานข้อมูล เก็บไว้ในตัวแปรชื่อ 
		\item บรรทัดที่ 6	เรียกใช้งานการจัดการ JSON web token เก็บไว้ในตัวแปรชื่อ jwt
		\item บรรทัดที่ 4	เรียกใช้งานการถอดรหัสรูปภาพ base64 เก็บไว้ในตัวแปรชื่อ decode64
		\item บรรทัดที่ 8-15	กำหนดค่าของเริ่มต้นของ Middleware
		\item บรรทัดที่ 16	กำหนดการ routing เพื่อตรวจสอบการร้องขอก่อนจะให้เรียกทรัพยากร ด้วยการตรวจสอบ JWT ที่ส่งมาด้วยกับ HTTP Request 
		\item บรรทัดที่ 18	กำหนดการ routing สำหรับการจ้องขอเพื่อขอ JWT 
		\item บรรทัดที่ 19	กำหนดการ routing สำหรับการตรวจสอบ JWT
		\item บรรทัดที่ 20	กำหนดการ routing สำหรับการร้องขอทรัพยากรรายการยาแบบละเอียด
		\item บรรทัดที่ 21	กำหนดการ routing สำหรับการร้องขอทรัพยากรแบบการค้นหายาทั่วไป
		\item บรรทัดที่ 22	กำหนดการ routing สำหรับการร้องขอทรัพยากรแบบการค้นหายาขั้นสูง
		\item บรรทัดที่ 23	กำหนดการ routing สำหรับการร้องขอทรัพยากรรายการรูปภาพของยา
		\item บรรทัดที่ 24	กำหนดการ routing สำหรับการร้องขอทรัพยากรรายการยาที่ผู้ใช้งานบุ๊กมาร์กไว้ที่เครื่อง
		\item บรรทัดที่ 25	กำหนดการ routing สำหรับการร้องขอทรัพยากรรายการ Dropdownlist ลักษณะสีของยา
		\item บรรทัดที่ 26	กำหนดการ routing สำหรับการร้องขอทรัพยากรรายการ Dropdownlist ประเภททะเบียนยา
		\item บรรทัดที่ 27	กำหนดการ routing สำหรับการร้องขอทรัพยากรรายการ Dropdownlist รูปแบบผลิตภัณฑ์
		\item บรรทัดที่ 28	กำหนดการ routing สำหรับการร้องขอทรัพยากรรายการ Dropdownlist สัญลักษณ์ที่ปรากฏบนเม็ดยา
		\item บรรทัดที่ 29	กำหนดการ routing สำหรับการร้องขอทรัพยากรรายการ Dropdownlist ขนาดของยา
		\item บรรทัดที่ 30	กำหนดการ routing สำหรับการร้องขอทรัพยากรรายการ Dropdownlist สถานะของยา
		\item บรรทัดที่ 31	กำหนดการ routing สำหรับการร้องขอทรัพยากรรายการ Dropdownlist รูปร่างลักษณะยา
		\item บรรทัดที่ 32	กำหนดการ routing สำหรับการร้องขอทรัพยากรรายการ Dropdownlist ของรูปทรงยา
		\item บรรทัดที่ 33	กำหนดการ routing สำหรับการร้องขอเพื่อส่งรูปภาพยามาที่เว็บเซอร์วิสเพื่อประมวลผลรูปภาพยา
		\item บรรทัดที่ 34-36	ให้เว็บเซอร์วิสทำงานที่ PORT ที่ถูกกำหนดไว้
	\end{itemize}
	

\section{การประมวลผลภาพ}
	การประมวลภาพรูปภาพใช้ไลบรารีของ nodejs ชื่อ opencv4nodejs 
	แสดงดังภาพที่ \ref{Fig:imageProcessing1}

	\begin{figure}[H]
		{\setstretch{1.0}\begin{lstlisting}\ContinueLineNumber
predict:   function(req,  res,  next)  {        
    let  testDataPath  =  `./src/assets/threshold/${req.query.filename}`;        
    let  img  =  cv.imread(testDataPath);        
    img  =  img.resize(64,  64);        
    try  {            
        const  desc  =  computeHOGDescriptorFromImage(img,  false);            
        if  (!desc)  {                
            throw  new  Error(`Computing HOG descriptor failed for file: ${file}`);            
        }            
        let  svmPredict  =  svm.predict(desc);            
        let  dshape  =   [];            
        let  predict  =   {                
            label:  lccs[svmPredict],  
            colors:  req.query.colors,  
			colorid:  req.query.colorid,  
			colorname:  req.query.colorname,  
			width:  req.query.width,  
			dshape:  lccs[svmPredict]]            
        }            
        return  res.status(200).send(predict);        
    }   
		\end{lstlisting}}
		\caption{การประมวลผลภาพ}
		\label{Fig:imageProcessing1}
	\end{figure}
	\begin{figure}[H]
		{\setstretch{1.0}\begin{lstlisting}\ContinueLineNumber
	catch  (err)  {            
		return  res.status(200).send({                
			isError:  true,  
			messages:  err            
		});        
	}},  getSizeAndColors:   function(req,  res,  next)  {        
	let  testDataPath  =  `./src/assets/images/${req.query.filename}`;        
	let  mat  =  cv.imread(testDataPath);        
	mat  =  mat.resize(1024,  1024);        
	const  know_width  =  11.3       
	const  know_heigth  =  7.3       
	let  boundingRect  =   [];        
	let  num  =  0;        
	const  thd  =  1500;        
	const  threshold  =  100;        
	const  matGray  =  mat.bgrToGray();        
	var  thresh  =  matGray.threshold(100,  255,  0);    
	const  contours  =  thresh.findContours(cv.RETR_CCOMP,  cv.CHAIN_APPROX_SIMPLE,  new  cv.Point(0,  0));        
try  {            
	for  (let  i  =  0;  i  <  contours.length;  i++)  {                
		var  brect  =  contours[i].boundingRect();                
		const  pt1  =  new  cv.Point(brect.x,  brect.y);                
		const  pt2  =  new  cv.Point(brect.x  +  brect.width,  brect.y  +  brect.height);                
		if  (brect.width  *  brect.height  >=  thd)  {                    
			boundingRect.push(brect);                    
			num++;                    
			mat.drawRectangle(pt2,  pt1,  blue);                    
			const  region  =  mat.getRegion(brect);               
		}   
		else  {                    
			mat.drawRectangle(pt2,  pt1,  red);                
		}            
	}           
	var  matThresh  =  mat.threshold(100,  255,  0);           
	\end{lstlisting}}
	\caption{การประมวลผลภาพ(ต่อ)}
	\label{Fig:imageProcessing2}
	\end{figure}

	\begin{figure}[H]
		{\setstretch{1.0}\begin{lstlisting}\ContinueLineNumber
	if  (num  <  2)  {                
		return  res.send({                    
			isError:  true,  
			messages:   'contounrs < 2.'                
		});            
	}            
	boundingRect.sort(function(rect1,  rect2)  {                
		if  (rect1.height  +  rect1.width  >  rect2.height  +  rect2.width)  {                    
			return  1;                
		}            
	});            
	const  padding  =  5;            
	const  newRect  =  new  cv.Rect((boundingRect[0].x  -  padding)  >=  0  ?  boundingRect[0].x  -  padding  :  0,   (boundingRect[0].y  -  padding)  >=  0  ?  boundingRect[0].y  -  padding  :  0,  mat.cols  -  (boundingRect[0].x  +  boundingRect[0].width  +  padding  *  2)  >=  0  ?  (boundingRect[0].width  +  padding  *  2)  :  boundingRect[0].width,  mat.rows  -  (boundingRect[0].y  +  boundingRect[0].height  +  padding  *  2)  >=  0  ?  (boundingRect[0].height  +  padding  *  2)  :  boundingRect[0].height);
	const  region  =  thresh.getRegion(newRect);            
	var  cols  =  region.cols;            
	var  rows  =  region.rows;            
	var  pad  =  0;            
	if  (region.cols  >  region.rows)  {                
		pad  =  region.cols  -  region.rows;                
		rows  +=  pad;            
	}   
	else  {                
		pad  =  region.rows  -  region.cols;                
		cols  +=  pad;            
	}            
	var  matPadded  =  new  cv.Mat(rows,  cols,  cv.CV_8UC1,  0);            
	for  (let  i  =  0;  i  <  region.rows;  i++)  {                
		for  (let  j  =  0;  j  <  region.cols;  j++)  {                    
			let  atMat  =  region.at(i,  j);                    
			if  (region.cols  >  region.rows)  {                        
				matPadded.set(i  +  (pad  /  2),  j,  atMat);                    
			}   
		\end{lstlisting}}
		\caption{การประมวลผลภาพ(ต่อ)}
		\label{Fig:imageProcessing2}
	\end{figure}

	\begin{figure}[H]
		{\setstretch{1.0}\begin{lstlisting}\ContinueLineNumber
			else  {                        
				matPadded.set(i,  j  +  (pad  /  2),  atMat);                    
			}                
		}            
	}            
	matPadded  =  matPadded.resize(64,  64);            
	let  matCanny  =  matPadded.canny(threshold,  threshold,  3,  true);            
	let  matInvert  =  matCanny.bitwiseNot();            
	cv.imwrite(`./src/assets/threshold/${req.query.filename}`,  matInvert);           
	var  point  =   [];            
	var  colors  =   [];            
		var  colorid  =   [];            
		var  colorname  =   [];            
	let  hexColors  =   ['FFFFFF',  'ea0000',  'fe700e',  'ffe400',  'efeddb',  'a8e26a',  '4a9fe0',  '120992',  'fdb1ef',  'c977db',  'c2924a',  'cccccc',  '000000']            
	let  nameColors  =   [];     
	point.push([boundingRect[0].x  +  (boundingRect[0].width  /  4),  boundingRect[0].y  +  (boundingRect[0].height  /  4)]);            
	point.push([boundingRect[0].x  +  (boundingRect[0].width  /  4),  boundingRect[0].y  +  (boundingRect[0].height  *  3  /  4)]);            
	point.push([boundingRect[0].x  +  (boundingRect[0].width  *  3  /  4),  boundingRect[0].y  +  (boundingRect[0].height  /  4)]);            
	point.push([boundingRect[0].x  +  (boundingRect[0].width  *  3  /  4),  boundingRect[0].y  +  (boundingRect[0].height  *  3  /  4)]);            
	for  (let  i  =  0;  i  <  4;  i++)  {                
		let  bgr  =  matThresh.atRaw(parseInt(point[i][1]),  parseInt(point[i][0]));                
		let  diffColor  =  101;                
		let  indexColor  =  0;                
		let  hexColor  =  rgbHex(bgr[2],  bgr[1],  bgr[0]); 
	 	\end{lstlisting}}
		\caption{การประมวลผลภาพ(ต่อ)}
		\label{Fig:imageProcessing3}
	\end{figure}
	\begin{figure}[H]
		{\setstretch{1.0}\begin{lstlisting}
		for  (let  j  =  0;  j  <  hexColors.length;  j++)  {                    
			let  tmp  =  cd.compare(hexColor,  hexColors[j]);                    
			if  (tmp  <  diffColor)  {                        
				diffColor  =  tmp;                        
				indexColor  =  j;                    
			}                
		}                
		if  (!colorid.includes(indexColor  +  1))  {                    
			colors.push(hexColors[indexColor]);                    
			colorid.push(indexColor  +  1);                    
			colorname.push(nameColors[indexColor]);                
		}            
	}            
	const  Rwidth  =  boundingRect[1].width  <  boundingRect[1].height  ?  boundingRect[1].height  :  boundingRect[1].width;            
	const  Owidth  =  boundingRect[0].width  <  boundingRect[0].height  ?  boundingRect[0].height  :  boundingRect[0].width;            
	const  width  =   (know_width  /  Rwidth)  *  Owidth;            
	req.query.colors  =  colors;            
	req.query.colorid  =  colorid;            
	req.query.colorname  =  colorname;            
	req.query.boundingRect  =  boundingRect[0];            
	req.query.width  =  width;            
	next();        
	}   
	catch  (err)  {            
		return  res.status(200).send({                
			isError:  true,  
			messages:  err            
		});        
	}    
}   
	 	\end{lstlisting}}
		\caption{การประมวลผลภาพ(ต่อ)}
		\label{Fig:imageProcessing4}
	\end{figure}
	จากภาพที่ \ref{Fig:imageProcessing1} อธิบายการทำงานของการประมวลผลภาพ ได้ดังนี้
	\begin{itemize}[label={--}]
		\item บรรทัดที่ 1-26	ฟังก์ชันการจำแนกรูปทรงของยา
		\item บรรทัดที่ 27-137 ฟังก์ชันสำหรับการประมวลผลภาพเพื่อหาขนาดและสีของเม็ดยา
		\item บรรทัดที่ 2-3	กำหนด path และอ่านไฟล์มาเก็บในตัวแปร img
		\item บรรทัดที่ 4	ปรับขนาดของรูปภาพเป็นขนาด 64x64 พิกเซล
		\item บรรทัดที่ 6-9	เรียกฟังก์ชันสำหรับการแปรงรูปภาพให้เป็นอาเรย์ เพื่อที่จะนำไปจำแนกรูปทรง
		\item บรรทัดที่ 10	เรียกฟังก์ชันเพื่อจำแนกรูปทรงของยาและเก็บที่ตัวแปรชื่อว่า svmPredict
		\item บรรทัดที่ 11	กำหนดตัวชื่อ dshape เป็น label ของรูปทรงต่างๆ 
		\item บรรทัดที่ 12-19	กำหนดตัวชื่อ predict สำหรับเก็บผลลัพธ์การประมวลผล
		\item บรรทัดที่ 20	คืนผลลัพธ์กลับไปที่เครื่องผู้ใช้งาน
		\item บรรทัดที่ 28-29 กำหนด path และอ่านไฟล์มาเก็บในตัวแปร mat
		\item บรรทัดที่ 31-32 กำหนดขนาดด้านยาวและด้านกว้างของวัตถุอ้างอิง ในตัวแปรชื่อ {know\_width} และ {know\_height}
		\item บรรทัดที่ 33	กำหนดตัวแปรชื่อว่า boundingRect สำหรับเก็บ Contounr
		\item บรรทัดที่ 35	กำหนดตัวแปรชื่อ thd เก็บค่า threshold เพื่อค่าขั้นต่ำของขนาด Contounr 
		\item บรรทัดที่ 36	กำหนดตัวแปรชื่อ threshold เก็บค่า threshold ของการปรับภาพให้เป็น Canny Edge
		\item บรรทัดที่ 37	ปรับภาพเป็นให้ภาพ Grayscale
		\item บรรทัดที่ 38	ปรับ threshold ของภาพ Grayscale
		\item บรรทัดที่ 39	นำภาพจากตัวแปร thresh มาหา Contounrs ด้วยฟังก์ชัน fildContounrs 
		\item บรรทัดที่ 41-53	ตรวจหา contounr ที่เป็นวัถตุอ้างอิงและวัตถุเป้าหมาย (ยา) และเก็บในอาเรย์ชื่อ boundingRect 
		\item บรรทัดที่ 55-59	ถ้าหากมี contounr เพียง 0 – 1 อัน จะไม่สามารถดำเนินการต่อไปได้ และคืนข้อผิดพลาดกลับไปที่เครื่องผู้ใช้งาน
		\item บรรทัดที่ 61-63	เป็นการเรียงลำดับขนาดพิ้นที่ของ contounr จากน้อยไปมาก 
		\item บรรทัดที่ 66-89 เป็นการตัดภาพเอาเฉพาะภาพยาเม็ด และปรับภาพให้เป็นสี่เหลี่ยมจัตุรัส
		\item บรรทัดที่ 91-93 นำภาพเม็ดยามาหา Canny Edge และเขียนไฟล์รูปภาพ
		\item บรรทัดที่ 94 	กำหนดตัวแปรชื่อ point สำหรับการเก็บตำแหน่งของสีของภาพเม็ดยา 
		\item บรรทัดที่ 95-97	กำหนดตัวแปรชื่อ colors colored และ colorname สำหรับเก็บข้อมูลของสี
		\item บรรทัดที่ 98	กำหนดตัวแปรชื่อ hexColors สำหรับเก็บค่าโค้ดสีจากฐานข้อมูล
		\item บรรทัดที่ 99	กำหนดตัวแปรชื่อ nameColors สำหรับชื่อสีจากฐานข้อมูล
		\item บรรทัดที่ 100-120 เป็นการดึงค่าสี BGR จากภาพยาเม็ดด้วยตำแหน่งที่ได้อาเรย์ชื่อ point และนำค่าสีไปเปรียบเทียบกับอาเรย์ชื่อ hexColor เพื่อหาค่าสีที่ใกล้เคียงกันมากที่สุด
		\item บรรทัดที่ 122-124 เป็นการวัดขนาดด้านยาวของเม็ดยา โดยใช้การอ้างอิงจากวัตถุที่รู้ขนาดอยู่แล้ว
		\item บรรทัดที่ 125-129 เก็บผลลัพธ์การพิสูจน์เอกลักษณ์ยาเม็ด ไว้ใน req.query 
		\item บรรทัดที่ 130	เรียกฟังก์ชัน next() เพื่อทำงานฟังก์ชันต่อไป
		\item บรรทัดที่ 131-135 ถ้าหากการประมวลผลภาพเกิดข้อผิดพลาด จะทำการส่งข้อความบอกข้อผิดพลาดไปที่เครื่องผู้ใช้งาน
	\end{itemize}

